\section{Appendix C - Tools Utilised}

    \subsection*{Nessus}
    Nessus is a vulnerability scanner developed by Tenable Network Security. Today Nessus is on of the most used vulnerability scanners used in the pentesting field, alongside OpenVAS and CoreImpact.\\
    In this penetration test, Nessus was used in the start of the project in order to discover vulnerable services and servers in the network.
     
    \subsection*{OpenVAS}
    OpenVAS is a framework containing several services and tools related to vulnerability scanning. OpenVAS is open-source and free, making this product very popular amongst penetration testers worldwide.
     
    \subsection*{Metasploit}
    Metasploit is a very popular, open-source, framework created by Rapid7. Metasploit Framework is a tool for developing and executing exploits against vulnerable services/hosts, and contains thousands of different exploits, targeting Windows, Linux, BSD etc.
     
    \subsection*{Hashcat}
    Hashcat is a free password cracker with multi-gpu support. The authors claim that hashcat is the worlds fastest password cracker.\\
    We used hashcat when cracking the password we found during our assessment of the network.
     
    \subsection*{HashID}
    Hashid is a tool used to identify hashes. We used this tool to identify the hashes we found during the penetration test, in order to crack and recover the relevant passwords.
     
    \subsection*{NBTScan}
    NBTScan is a command line tool used to scan for NETBIOS nameservers on a network (local or remote). We used this tool in order to find open shares on the network.
     
    \subsection*{Dsniff}
    dsniff is a command line tool used to sniff passwords on a network. This tool handles everything from FTP and Telnet, to HTTP and RIP. This tool was used when we discovered the password transmitted over the Telnet protocol.
     
    \subsection*{Wireshark}
    This free and open-source packet analyzer is mainly used for network troubleshooting and network analysis. Wireshark can sniff and analyze about any protocol used to transfer data over a network, and this information can later be dumped and saved to a .pcap file for later analysis.\\
    This tool was used to sniff data going over the network and between the computers we did a penetration test on, in hope of finding anything that could be exploited.
     
    \subsection*{NMAP}
    NMAP (Network Mapper) is a network scanner used to discover hosts and services on a network, and thus create a map of the network. NMAP was used in conjunction with Wireshark to map out the entire network, and collect information about the hosts and services running on the network.
     
    \subsection*{Python}
    Python is a very poopular general-purpose programming language with a huge emphasis on code readability. We used this programming language to try out some exploits against the SMB protocol and a web server vulnerability.
     
    \subsection*{find}
    find is a built-in command in linux, used to find files.\\
    We used this command to find files with the setuid bit set, with the hope that one of those files could lead to command execution with root privileges.\\
   \vspace{-0.5cm}\begin{lstlisting}
find directory -user root -perm -4000 -exec ls -ldb {} \; >/tmp/filename
\end{lstlisting}

