\section{Appendix D - Risk Calculations}

In order to categorize the vulnerabilites that are found, we have to define the categorizing measure. \\
In the penetration test has catogorized the vulnerabilites by risk. \\
The risk is calculated by multiplying the impact (damage done) and Threat (chance of impact). \(Risk = Imapct * Threat\) \\
\begin{table}[h]
\centering
\begin{tabular}{llllll}
\multirow{6}{*}{Threat} & \multicolumn{5}{c}{Impact}                                                                                                                                             \\ \cline{3-6} 
                        & \multicolumn{1}{l|}{}             & \multicolumn{1}{l|}{Low (1)} & \multicolumn{1}{l|}{Medium (2)} & \multicolumn{1}{l|}{High (3)} & \multicolumn{1}{l|}{Critical (4)} \\ \cline{2-6} 
                        & \multicolumn{1}{l|}{Low (1)}      & \multicolumn{1}{l|}{1}       & \multicolumn{1}{l|}{2}          & \multicolumn{1}{l|}{3}        & \multicolumn{1}{l|}{4}            \\ \cline{2-6} 
                        & \multicolumn{1}{l|}{Medium (2)}   & \multicolumn{1}{l|}{2}       & \multicolumn{1}{l|}{4}          & \multicolumn{1}{l|}{6}        & \multicolumn{1}{l|}{8}            \\ \cline{2-6} 
                        & \multicolumn{1}{l|}{High (3)}     & \multicolumn{1}{l|}{3}       & \multicolumn{1}{l|}{6}          & \multicolumn{1}{l|}{9}        & \multicolumn{1}{l|}{12}           \\ \cline{2-6} 
                        & \multicolumn{1}{l|}{Critical (4)} & \multicolumn{1}{l|}{4}       & \multicolumn{1}{l|}{8}          & \multicolumn{1}{l|}{12}       & \multicolumn{1}{l|}{16}           \\ \cline{2-6} 
\end{tabular}
\caption{Risk Table}
\label{my-label}
\end{table}

For the risk score to be used practically, we have to compress them into smaller categories as seen in table 2.

\begin{table}[h]
\centering
\begin{tabular}{|l|l|l|}
\hline
Low & Medium & High \\ \hline
1-4 & 5-8    & 9-16 \\ \hline
\end{tabular}
\caption{Risk categorization}
\label{my-label}
\end{table}
