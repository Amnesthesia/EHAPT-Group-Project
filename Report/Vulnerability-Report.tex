\documentclass{article}
\pagestyle{empty}

%\usepackage{color}
\usepackage{tabularx}
\usepackage{geometry}
\usepackage{comment}
\usepackage{longtable}
\usepackage{titlesec}
\usepackage{chngpage}
\usepackage{calc}
\usepackage{url}
\usepackage[utf8x]{inputenc}

\DeclareUnicodeCharacter {135}{{\textascii ?}}
\DeclareUnicodeCharacter {129}{{\textascii ?}}
\DeclareUnicodeCharacter {128}{{\textascii ?}}

\usepackage{colortbl}

% must come last
\usepackage{hyperref}
\definecolor{linkblue}{rgb}{0.11,0.56,1}
\definecolor{inactive}{rgb}{0.56,0.56,0.56}
\definecolor{openvas_debug}{rgb}{0.78,0.78,0.78}
\definecolor{openvas_false_positive}{rgb}{0.2275,0.2275,0.2275}
\definecolor{openvas_log}{rgb}{0.2275,0.2275,0.2275}
\definecolor{openvas_hole}{rgb}{0.7960,0.1137,0.0902}
\definecolor{openvas_note}{rgb}{0.3255,0.6157,0.7961}
\definecolor{openvas_report}{rgb}{0.68,0.74,0.88}
\definecolor{openvas_user_note}{rgb}{1.0,1.0,0.5625}
\definecolor{openvas_user_override}{rgb}{1.0,1.0,0.5625}
\definecolor{openvas_warning}{rgb}{0.9764,0.6235,0.1922}
\definecolor{chunk}{rgb}{0.9412,0.8275,1}
\definecolor{line_new}{rgb}{0.89,1,0.89}
\definecolor{line_gone}{rgb}{1.0,0.89,0.89}
\hypersetup{colorlinks=true,linkcolor=linkblue,urlcolor=blue,bookmarks=true,bookmarksopen=true}
\usepackage[all]{hypcap}

%\geometry{verbose,a4paper,tmargin=24mm,bottom=24mm}
\geometry{verbose,a4paper}
\setlength{\parskip}{\smallskipamount}
\setlength{\parindent}{0pt}
\title{Scan Report}

\pagestyle{headings}
\pagenumbering{arabic}

\begin{document}

\maketitle

\renewcommand{\abstractname}{Summary}
\begin{abstract}
This document reports on the results of an automatic security scan.
The scan started at Mon Oct 27 02:13:47 2014 UTC and ended at Mon Oct 27 03:05:42 2014 UTC.  The
report first summarises the results found.  Then, for each host,
the report describes every issue found.  Please consider the
advice given in each description, in order to rectify the issue.
\end{abstract}
\tableofcontents
\newpage
\section{Result Overview}

\begin{longtable}{|l|l|l|l|l|l|l|}
\hline
\rowcolor{openvas_report}Host&Most Severe Result(s)&High&Medium&Low&Log&False Positives\\
\hline
\endfirsthead
\multicolumn{6}{l}{\hfill\ldots (continued) \ldots}\\
\hline
\rowcolor{openvas_report}Host&Most Severe Result(s)&High&Medium&Low&Log&False Positives\\
\hline
\endhead
\hline
\multicolumn{6}{l}{\ldots (continues) \ldots}\\
\endfoot
\hline
\endlastfoot
\hline
\hyperref[host:192.168.248.129]{192.168.248.129}&Severity: High&1&1&0&0&0\\
\hline
\hyperref[host:192.168.248.130]{192.168.248.130}&Severity: Medium&0&3&0&0&0\\
\hline
\hyperref[host:192.168.248.131]{192.168.248.131 (LH-0XZPOTYEAFVA)}&Severity: High&1&4&0&0&0\\
\hline
\hline
Total: 3&&2&8&0&0&0\\
\hline
\end{longtable}
Vendor security updates are not trusted.\\
Overrides are on.  When a result has an override, this report uses the threat of the override.\\
Notes are included in the report.\\
This report might not show details of all issues that were found.\\
It only lists hosts that produced issues.\\
Issues with the threat level "Low" are not shown.\\
Issues with the threat level "Log" are not shown.\\
Issues with the threat level "Debug" are not shown.\\
Issues with the threat level "False Positive" are not shown.\\
\\
This report contains all 10 results selected by the filtering described above.  Before filtering there were 77 results.\section{Results per Host}

\subsection{Ubuntu}
\label{host:192.168.248.129}

\begin{tabular}{ll}
Host scan start&Mon Oct 27 02:14:02 2014 UTC\\
Host scan end&Mon Oct 27 02:58:57 2014 UTC\\
\end{tabular}

\begin{longtable}{|l|l|}
\hline
\rowcolor{openvas_report}Service (Port)&Threat Level\\
\hline
\endfirsthead
\multicolumn{2}{l}{\hfill\ldots (continued) \ldots}\\
\hline
\rowcolor{openvas_report}Service (Port)&Threat Level\\
\hline
\endhead
\hline
\multicolumn{2}{l}{\ldots (continues) \ldots}\\
\endfoot
\hline
\endlastfoot
\hline
\hyperref[port:192.168.248.129 http (80/tcp) High]{http (80/tcp)}&High\\
\hline
\hyperref[port:192.168.248.129 general/tcp Medium]{general/tcp}&Medium\\
\hline
\end{longtable}


%\subsection*{Security Issues and Fixes -- 192.168.248.129}

\subsubsection{High http (80/tcp)}
\label{port:192.168.248.129 http (80/tcp) High}

\begin{longtable}{|p{\textwidth * 1}|}
\hline
\rowcolor{openvas_hole}{\color{white}{High (CVSS: 7.5) }}\\
\rowcolor{openvas_hole}{\color{white}{NVT: Lighttpd Multiple vulnerabilities}}\\
\hline
\endfirsthead
\hfill\ldots continued from previous page \ldots \\
\hline
\endhead
\hline
\ldots continues on next page \ldots \\
\endfoot
\hline
\endlastfoot
\\
\rowcolor{white}{\verb==}\\
\rowcolor{white}{\verb==}\\
\\
OID of test routine: 1.3.6.1.4.1.25623.1.0.802072\\
\\

      \hline
      \\
\textbf{References}\\
\rowcolor{white}{\verb=CVE: CVE-2014-2323, CVE-2014-2324=}\\
\rowcolor{white}{\verb=BID:66153, 66157=}\\
\rowcolor{white}{\verb=Other:=}\\
\rowcolor{white}{\verb=  URL:http://osvdb.org/104381=}\\
\rowcolor{white}{\verb=   URL:http://osvdb.org/104382=}\\
\rowcolor{white}{\verb=   URL:http://seclists.org/oss-sec/2014/q1/561=}\\
\rowcolor{white}{\verb=   URL:http://download.lighttpd.net/lighttpd/security/lighttpd_sa_2014_01.txt=}\\
\end{longtable}

\begin{footnotesize}\hyperref[host:192.168.248.129]{[ return to 192.168.248.129 ]}\end{footnotesize}
\subsubsection{Medium general/tcp}
\label{port:192.168.248.129 general/tcp Medium}

\begin{longtable}{|p{\textwidth * 1}|}
\hline
\rowcolor{openvas_warning}{\color{white}{Medium (CVSS: 2.6) }}\\
\rowcolor{openvas_warning}{\color{white}{NVT: TCP timestamps}}\\
\hline
\endfirsthead
\hfill\ldots continued from previous page \ldots \\
\hline
\endhead
\hline
\ldots continues on next page \ldots \\
\endfoot
\hline
\endlastfoot
\\
\rowcolor{white}{\verb=It was detected that the host implements RFC1323.=}\\
\rowcolor{white}{\verb=The following timestamps were retrieved with a delay of 1 seconds in-between:=}\\
\rowcolor{white}{\verb=Paket 1: 1776245=}\\
\rowcolor{white}{\verb=Paket 2: 1776498=}\\
\rowcolor{white}{\verb==}\\
\rowcolor{white}{\verb==}\\
\\
OID of test routine: 1.3.6.1.4.1.25623.1.0.80091\\
\\

      \hline
      \\
\textbf{References}\\
\rowcolor{white}{\verb=Other:=}\\
\rowcolor{white}{\verb=  URL:http://www.ietf.org/rfc/rfc1323.txt=}\\
\end{longtable}

\begin{footnotesize}\hyperref[host:192.168.248.129]{[ return to 192.168.248.129 ]}\end{footnotesize}
\subsection{OpenBSD}
\label{host:192.168.248.130}

\begin{tabular}{ll}
Host scan start&Mon Oct 27 02:14:02 2014 UTC\\
Host scan end&Mon Oct 27 02:33:08 2014 UTC\\
\end{tabular}

\begin{longtable}{|l|l|}
\hline
\rowcolor{openvas_report}Service (Port)&Threat Level\\
\hline
\endfirsthead
\multicolumn{2}{l}{\hfill\ldots (continued) \ldots}\\
\hline
\rowcolor{openvas_report}Service (Port)&Threat Level\\
\hline
\endhead
\hline
\multicolumn{2}{l}{\ldots (continues) \ldots}\\
\endfoot
\hline
\endlastfoot
\hline
\hyperref[port:192.168.248.130 general/tcp Medium]{general/tcp}&Medium\\
\hline
\hyperref[port:192.168.248.130 ssh (22/tcp) Medium]{ssh (22/tcp)}&Medium\\
\hline
\end{longtable}


%\subsection*{Security Issues and Fixes -- 192.168.248.130}

\subsubsection{Medium general/tcp}
\label{port:192.168.248.130 general/tcp Medium}

\begin{longtable}{|p{\textwidth * 1}|}
\hline
\rowcolor{openvas_warning}{\color{white}{Medium (CVSS: 2.6) }}\\
\rowcolor{openvas_warning}{\color{white}{NVT: TCP timestamps}}\\
\hline
\endfirsthead
\hfill\ldots continued from previous page \ldots \\
\hline
\endhead
\hline
\ldots continues on next page \ldots \\
\endfoot
\hline
\endlastfoot
\\
\rowcolor{white}{\verb=It was detected that the host implements RFC1323.=}\\
\rowcolor{white}{\verb=The following timestamps were retrieved with a delay of 1 seconds in-between:=}\\
\rowcolor{white}{\verb=Paket 1: 721791790=}\\
\rowcolor{white}{\verb=Paket 2: -351001889=}\\
\rowcolor{white}{\verb==}\\
\rowcolor{white}{\verb==}\\
\\
OID of test routine: 1.3.6.1.4.1.25623.1.0.80091\\
\\

      \hline
      \\
\textbf{References}\\
\rowcolor{white}{\verb=Other:=}\\
\rowcolor{white}{\verb=  URL:http://www.ietf.org/rfc/rfc1323.txt=}\\
\end{longtable}

\begin{footnotesize}\hyperref[host:192.168.248.130]{[ return to 192.168.248.130 ]}\end{footnotesize}
\subsubsection{Medium ssh (22/tcp)}
\label{port:192.168.248.130 ssh (22/tcp) Medium}

\begin{longtable}{|p{\textwidth * 1}|}
\hline
\rowcolor{openvas_warning}{\color{white}{Medium (CVSS: 5.0) }}\\
\rowcolor{openvas_warning}{\color{white}{NVT: OpenSSH Legacy Certificate Signing Information Disclosure Vulnerability}}\\
\hline
\endfirsthead
\hfill\ldots continued from previous page \ldots \\
\hline
\endhead
\hline
\ldots continues on next page \ldots \\
\endfoot
\hline
\endlastfoot
\\
\rowcolor{white}{\verb= Summary:=}\\
\rowcolor{white}{\verb= Checks whether OpenSSH is prone to an information-disclosure vulnerability.=}\\
\rowcolor{white}{\verb=Successful exploits will allow attackers to gain access to sensitive=}\\
\rowcolor{white}{\verb=information; this may lead to further attacks.=}\\
\rowcolor{white}{\verb=Versions 5.6 and 5.7 of OpenSSH are vulnerable.=}\\
\rowcolor{white}{\verb= Vulnerability Detection:=}\\
\rowcolor{white}{\verb= The SSH banner is analysed for presence of openssh and the version=}\\
\rowcolor{white}{\verb=information is then taken from that banner.=}\\
\rowcolor{white}{\verb= Solution:=}\\
\rowcolor{white}{\verb= Updates are available. Please see the references for more information.=}\\
\rowcolor{white}{\verb==}\\
\rowcolor{white}{\verb==}\\
\\
OID of test routine: 1.3.6.1.4.1.25623.1.0.103064\\
\\

      \hline
      \\
\textbf{References}\\
\rowcolor{white}{\verb=CVE: CVE-2011-0539=}\\
\rowcolor{white}{\verb=BID:46155=}\\
\rowcolor{white}{\verb=Other:=}\\
\rowcolor{white}{\verb=  URL:https://www.securityfocus.com/bid/46155=}\\
\rowcolor{white}{\verb=   URL:http://www.openssh.com/txt/release-5.8=}\\
\rowcolor{white}{\verb=   URL:http://www.openssh.com=}\\
\end{longtable}

\begin{longtable}{|p{\textwidth * 1}|}
\hline
\rowcolor{openvas_warning}{\color{white}{Medium (CVSS: 3.5) }}\\
\rowcolor{openvas_warning}{\color{white}{NVT: openssh-server Forced Command Handling Information Disclosure Vulnerability}}\\
\hline
\endfirsthead
\hfill\ldots continued from previous page \ldots \\
\hline
\endhead
\hline
\ldots continues on next page \ldots \\
\endfoot
\hline
\endlastfoot
\\
\rowcolor{white}{\verb=According to its banner, the version of OpenSSH installed on the remote=}\\
\rowcolor{white}{\verb=host is older than 5.7:=}\\
\rowcolor{white}{\verb= ssh-2.0-openssh_5.6=}\\
\rowcolor{white}{\verb= Summary:=}\\
\rowcolor{white}{\verb= The auth_parse_options function in auth-options.c in sshd in OpenSSH before 5.7=}\\
\rowcolor{white}{\verb=provides debug messages containing authorized_keys command options, which allows=}\\
\rowcolor{white}{\verb=remote authenticated users to obtain potentially sensitive information by=}\\
\rowcolor{white}{\verb=reading these messages, as demonstrated by the shared user account required by=}\\
\rowcolor{white}{\verb=Gitolite. NOTE: this can cross privilege boundaries because a user account may=}\\
\rowcolor{white}{\verb=intentionally have no shell or filesystem access, and therefore may have no=}\\
\rowcolor{white}{\verb=supported way to read an authorized_keys file in its own home directory.=}\\
\rowcolor{white}{\verb=OpenSSH before 5.7 is affected;=}\\
\rowcolor{white}{\verb= Solution:=}\\
\rowcolor{white}{\verb= Updates are available. Please see the references for more information.=}\\
\rowcolor{white}{\verb==}\\
\rowcolor{white}{\verb==}\\
\\
OID of test routine: 1.3.6.1.4.1.25623.1.0.103503\\
\\

      \hline
      \\
\textbf{References}\\
\rowcolor{white}{\verb=CVE: CVE-2012-0814=}\\
\rowcolor{white}{\verb=BID:51702=}\\
\rowcolor{white}{\verb=Other:=}\\
\rowcolor{white}{\verb=  URL:http://www.securityfocus.com/bid/51702=}\\
\rowcolor{white}{\verb=   URL:http://bugs.debian.org/cgi-bin/bugreport.cgi?bug=\verb-=-\verb=657445=}\\
\rowcolor{white}{\verb=   URL:http://packages.debian.org/squeeze/openssh-server=}\\
\rowcolor{white}{\verb=   URL:https://downloads.avaya.com/css/P8/documents/100161262=}\\
\end{longtable}

\begin{footnotesize}\hyperref[host:192.168.248.130]{[ return to 192.168.248.130 ]}\end{footnotesize}
\subsection{Windows Vista Business (6000)}
\label{host:192.168.248.131}

\begin{tabular}{ll}
Host scan start&Mon Oct 27 02:14:02 2014 UTC\\
Host scan end&Mon Oct 27 03:05:40 2014 UTC\\
\end{tabular}

\begin{longtable}{|l|l|}
\hline
\rowcolor{openvas_report}Service (Port)&Threat Level\\
\hline
\endfirsthead
\multicolumn{2}{l}{\hfill\ldots (continued) \ldots}\\
\hline
\rowcolor{openvas_report}Service (Port)&Threat Level\\
\hline
\endhead
\hline
\multicolumn{2}{l}{\ldots (continues) \ldots}\\
\endfoot
\hline
\endlastfoot
\hline
\hyperref[port:192.168.248.131 microsoft-ds (445/tcp) High]{microsoft-ds (445/tcp)}&High\\
\hline
\hyperref[port:192.168.248.131 epmap (135/tcp) Medium]{epmap (135/tcp)}&Medium\\
\hline
\hyperref[port:192.168.248.131 general/tcp Medium]{general/tcp}&Medium\\
\hline
\end{longtable}


%\subsection*{Security Issues and Fixes -- 192.168.248.131}

\subsubsection{High microsoft-ds (445/tcp)}
\label{port:192.168.248.131 microsoft-ds (445/tcp) High}

\begin{longtable}{|p{\textwidth * 1}|}
\hline
\rowcolor{openvas_hole}{\color{white}{High (CVSS: 10.0) }}\\
\rowcolor{openvas_hole}{\color{white}{NVT: Microsoft Windows SMB Server NTLM Multiple Vulnerabilities (971468)}}\\
\hline
\endfirsthead
\hfill\ldots continued from previous page \ldots \\
\hline
\endhead
\hline
\ldots continues on next page \ldots \\
\endfoot
\hline
\endlastfoot
\\
\rowcolor{white}{\verb=  Summary:=}\\
\rowcolor{white}{\verb=  This host is missing a critical security update according to=}\\
\rowcolor{white}{\verb=  Microsoft Bulletin MS10-012.=}\\
\rowcolor{white}{\verb=  Vulnerability Insight:=}\\
\rowcolor{white}{\verb=  - An input validation error exists while processing SMB requests and can=}\\
\rowcolor{white}{\verb=    be exploited to cause a buffer overflow via a specially crafted SMB packet.=}\\
\rowcolor{white}{\verb=  - An error exists in the SMB implementation while parsing SMB packets during=}\\
\rowcolor{white}{\verb=    the Negotiate phase causing memory corruption via a specially crafted SMB=}\\
\rowcolor{white}{\verb=    packet.=}\\
\rowcolor{white}{\verb=  - NULL pointer dereference error exists in SMB while verifying the 'share'=}\\
\rowcolor{white}{\verb=    and 'servername' fields in SMB packets causing denial of service.=}\\
\rowcolor{white}{\verb=  - A lack of cryptographic entropy when the SMB server generates challenges=}\\
\rowcolor{white}{\verb=    during SMB NTLM authentication and can be exploited to bypass the=}\\
\rowcolor{white}{\verb=    authentication mechanism.=}\\
\rowcolor{white}{\verb=  Impact:=}\\
\rowcolor{white}{\verb=  Successful exploitation will allow remote attackers to execute arbitrary=}\\
\rowcolor{white}{\verb=  code or cause a denial of service or bypass the authentication mechanism=}\\
\rowcolor{white}{\verb=  via brute force technique.=}\\
\rowcolor{white}{\verb=  Impact Level: System/Application=}\\
\rowcolor{white}{\verb=  Affected Software/OS:=}\\
\rowcolor{white}{\verb=  Microsoft Windows 7=}\\
\rowcolor{white}{\verb=  Microsoft Windows 2000 Service Pack and prior=}\\
\rowcolor{white}{\verb=  Microsoft Windows XP Service Pack 3 and prior=}\\
\rowcolor{white}{\verb=  Microsoft Windows Vista Service Pack 2 and prior=}\\
\rowcolor{white}{\verb=  Microsoft Windows Server 2003 Service Pack 2 and prior=}\\
\rowcolor{white}{\verb=  Microsoft Windows Server 2008 Service Pack 2 and prior=}\\
\rowcolor{white}{\verb=  Solution:=}\\
\rowcolor{white}{\verb=  Run Windows Update and update the listed hotfixes or download and=}\\
\rowcolor{white}{\verb=  update mentioned hotfixes in the advisory from the below link,=}\\
\rowcolor{white}{\verb=  http://www.microsoft.com/technet/security/bulletin/ms10-012.mspx=}\\
\rowcolor{white}{\verb==}\\
\rowcolor{white}{\verb==}\\
\\
OID of test routine: 1.3.6.1.4.1.25623.1.0.902269\\
\\

      \hline
      \\
\textbf{References}\\
\rowcolor{white}{\verb=CVE: CVE-2010-0020, CVE-2010-0021, CVE-2010-0022, CVE-2010-0231=}\\
\rowcolor{white}{\verb=Other:=}\\
\rowcolor{white}{\verb=  URL:http://secunia.com/advisories/38510/=}\\
\rowcolor{white}{\verb=   URL:http://support.microsoft.com/kb/971468=}\\
\rowcolor{white}{\verb=   URL:http://www.vupen.com/english/advisories/2010/0345=}\\
\rowcolor{white}{\verb=   URL:http://www.microsoft.com/technet/security/bulletin/ms10-012.mspx=}\\
\end{longtable}

\begin{footnotesize}\hyperref[host:192.168.248.131]{[ return to 192.168.248.131 ]}\end{footnotesize}
\subsubsection{Medium epmap (135/tcp)}
\label{port:192.168.248.131 epmap (135/tcp) Medium}

\begin{longtable}{|p{\textwidth * 1}|}
\hline
\rowcolor{openvas_warning}{\color{white}{Medium (CVSS: 5.0) }}\\
\rowcolor{openvas_warning}{\color{white}{NVT: DCE Services Enumeration}}\\
\hline
\endfirsthead
\hfill\ldots continued from previous page \ldots \\
\hline
\endhead
\hline
\ldots continues on next page \ldots \\
\endfoot
\hline
\endlastfoot
\\
\rowcolor{white}{\verb=  Summary:=}\\
\rowcolor{white}{\verb=  Distributed Computing Environment (DCE) services running on the remote host =}\\
\rowcolor{white}{\verb=can be enumerated by connecting on port 135 and doing the appropriate queries. =}\\
\rowcolor{white}{\verb=An attacker may use this fact to gain more knowledge=}\\
\rowcolor{white}{\verb=about the remote host.=}\\
\rowcolor{white}{\verb=  Solution:=}\\
\rowcolor{white}{\verb=  filter incoming traffic to this port.=}\\
\rowcolor{white}{\verb==}\\
\rowcolor{white}{\verb==}\\
\\
OID of test routine: 1.3.6.1.4.1.25623.1.0.10736\\
\end{longtable}

\begin{longtable}{|p{\textwidth * 1}|}
\hline
\rowcolor{openvas_warning}{\color{white}{Medium (CVSS: 5.0) }}\\
\rowcolor{openvas_warning}{\color{white}{NVT: DCE Services Enumeration}}\\
\hline
\endfirsthead
\hfill\ldots continued from previous page \ldots \\
\hline
\endhead
\hline
\ldots continues on next page \ldots \\
\endfoot
\hline
\endlastfoot
\\
\rowcolor{white}{\verb=Distributed Computing Environment (DCE) services running on the remote host=}\\
\rowcolor{white}{\verb=can be enumerated by connecting on port 135 and doing the appropriate queries.=}\\
\rowcolor{white}{\verb=An attacker may use this fact to gain more knowledge=}\\
\rowcolor{white}{\verb=about the remote host.=}\\
\rowcolor{white}{\verb=Here is the list of DCE services running on this host:=}\\
\rowcolor{white}{\verb=Port: 49152/tcp=}\\
\rowcolor{white}{\verb=     UUID: d95afe70-a6d5-4259-822e-2c84da1ddb0d, version 1=}\\
\rowcolor{white}{\verb=     Endpoint: ncacn_ip_tcp:192.168.248.131[49152]=}\\
\rowcolor{white}{\verb=Port: 49153/tcp=}\\
\rowcolor{white}{\verb=     UUID: f6beaff7-1e19-4fbb-9f8f-b89e2018337c, version 1=}\\
\rowcolor{white}{\verb=     Endpoint: ncacn_ip_tcp:192.168.248.131[49153]=}\\
\rowcolor{white}{\verb=     Annotation: Event log TCPIP=}\\
\rowcolor{white}{\verb=     UUID: 3c4728c5-f0ab-448b-bda1-6ce01eb0a6d5, version 1=}\\
\rowcolor{white}{\verb=     Endpoint: ncacn_ip_tcp:192.168.248.131[49153]=}\\
\rowcolor{white}{\verb=     Annotation: DHCP Client LRPC Endpoint=}\\
\rowcolor{white}{\verb=     UUID: 3c4728c5-f0ab-448b-bda1-6ce01eb0a6d6, version 1=}\\
\rowcolor{white}{\verb=     Endpoint: ncacn_ip_tcp:192.168.248.131[49153]=}\\
\rowcolor{white}{\verb=     Annotation: DHCPv6 Client LRPC Endpoint=}\\
\rowcolor{white}{\verb=     UUID: 06bba54a-be05-49f9-b0a0-30f790261023, version 1=}\\
\rowcolor{white}{\verb=     Endpoint: ncacn_ip_tcp:192.168.248.131[49153]=}\\
\rowcolor{white}{\verb=     Annotation: Security Center=}\\
\rowcolor{white}{\verb=Port: 49154/tcp=}\\
\rowcolor{white}{\verb=     UUID: 7ea70bcf-48af-4f6a-8968-6a440754d5fa, version 1=}\\
\rowcolor{white}{\verb=     Endpoint: ncacn_ip_tcp:192.168.248.131[49154]=}\\
\rowcolor{white}{\verb=     Annotation: NSI server endpoint=}\\
\rowcolor{white}{\verb=     UUID: 4b112204-0e19-11d3-b42b-0000f81feb9f, version 1=}\\
\rowcolor{white}{\verb=     Endpoint: ncacn_ip_tcp:192.168.248.131[49154]=}\\
\rowcolor{white}{\verb=     Named pipe : ssdpsrv=}\\
\rowcolor{white}{\verb=     Win32 service or process : ssdpsrv=}\\
\rowcolor{white}{\verb=     Description : SSDP service=}\\
\rowcolor{white}{\verb=Port: 49155/tcp=}\\
\rowcolor{white}{\verb=     UUID: 86d35949-83c9-4044-b424-db363231fd0c, version 1=}\\
\rowcolor{white}{\verb=     Endpoint: ncacn_ip_tcp:192.168.248.131[49155]=}\\
\rowcolor{white}{\verb=     UUID: a398e520-d59a-4bdd-aa7a-3c1e0303a511, version 1=}\\
\rowcolor{white}{\verb=     Endpoint: ncacn_ip_tcp:192.168.248.131[49155]=}\\
\rowcolor{white}{\verb=     Annotation: IKE/Authip API=}\\
\rowcolor{white}{\verb=Port: 49156/tcp=}\\
\rowcolor{white}{\verb=     UUID: 12345778-1234-abcd-ef00-0123456789ac, version 1=}\\
\rowcolor{white}{\verb=     Endpoint: ncacn_ip_tcp:192.168.248.131[49156]=}\\
\rowcolor{white}{\verb=     Named pipe : lsass=}\\
\rowcolor{white}{\verb=     Win32 service or process : lsass.exe=}\\
\rowcolor{white}{\verb=     Description : SAM access=}\\
\rowcolor{white}{\verb=Port: 49157/tcp=}\\
\rowcolor{white}{\verb=     UUID: 367abb81-9844-35f1-ad32-98f038001003, version 2=}\\
\rowcolor{white}{\verb=     Endpoint: ncacn_ip_tcp:192.168.248.131[49157]=}\\
\rowcolor{white}{\verb=Solution : filter incoming traffic to this port(s).=}\\
\rowcolor{white}{\verb==}\\
\rowcolor{white}{\verb==}\\
\\
OID of test routine: 1.3.6.1.4.1.25623.1.0.10736\\
\end{longtable}

\begin{footnotesize}\hyperref[host:192.168.248.131]{[ return to 192.168.248.131 ]}\end{footnotesize}
\subsubsection{Medium general/tcp}
\label{port:192.168.248.131 general/tcp Medium}

\begin{longtable}{|p{\textwidth * 1}|}
\hline
\rowcolor{openvas_warning}{\color{white}{Medium (CVSS: 3.3) }}\\
\rowcolor{openvas_warning}{\color{white}{NVT: Source routed packets}}\\
\hline
\endfirsthead
\hfill\ldots continued from previous page \ldots \\
\hline
\endhead
\hline
\ldots continues on next page \ldots \\
\endfoot
\hline
\endlastfoot
\\
\rowcolor{white}{\verb= Summary:=}\\
\rowcolor{white}{\verb= The remote host accepts loose source routed IP packets.=}\\
\rowcolor{white}{\verb=The feature was designed for testing purpose.=}\\
\rowcolor{white}{\verb=An attacker may use it to circumvent poorly designed IP filtering =}\\
\rowcolor{white}{\verb=and exploit another flaw. However, it is not dangerous by itself.=}\\
\rowcolor{white}{\verb= Solution:=}\\
\rowcolor{white}{\verb= drop source routed packets on this host or on other ingress=}\\
\rowcolor{white}{\verb=routers or firewalls.=}\\
\rowcolor{white}{\verb==}\\
\rowcolor{white}{\verb==}\\
\\
OID of test routine: 1.3.6.1.4.1.25623.1.0.11834\\
\end{longtable}

\begin{longtable}{|p{\textwidth * 1}|}
\hline
\rowcolor{openvas_warning}{\color{white}{Medium (CVSS: 2.6) }}\\
\rowcolor{openvas_warning}{\color{white}{NVT: TCP timestamps}}\\
\hline
\endfirsthead
\hfill\ldots continued from previous page \ldots \\
\hline
\endhead
\hline
\ldots continues on next page \ldots \\
\endfoot
\hline
\endlastfoot
\\
\rowcolor{white}{\verb=It was detected that the host implements RFC1323.=}\\
\rowcolor{white}{\verb=The following timestamps were retrieved with a delay of 1 seconds in-between:=}\\
\rowcolor{white}{\verb=Paket 1: 831744=}\\
\rowcolor{white}{\verb=Paket 2: 831845=}\\
\rowcolor{white}{\verb==}\\
\rowcolor{white}{\verb==}\\
\\
OID of test routine: 1.3.6.1.4.1.25623.1.0.80091\\
\\

      \hline
      \\
\textbf{References}\\
\rowcolor{white}{\verb=Other:=}\\
\rowcolor{white}{\verb=  URL:http://www.ietf.org/rfc/rfc1323.txt=}\\
\end{longtable}

\begin{footnotesize}\hyperref[host:192.168.248.131]{[ return to 192.168.248.131 ]}\end{footnotesize}

\begin{center}
\medskip
\rule{\textwidth}{0.1pt}

This file was automatically generated.
\end{center}

\end{document}